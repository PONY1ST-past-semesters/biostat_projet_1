\documentclass[../main.tex]{subfiles}

\begin{document}
\section*{Exercice 1a}
\begin{CJK*}{UTF8}{gbsn}
Démontrez que la loi de Poisson appartient à la famille exponentielle sous la mesure de comptage.

\smallskip
\paragraph{Solution}
Soit $\nu$ la mesure de comptage supportée sur $\mathbb{N}$.
La loi de Poisson avec paramètre $\lambda > 0$ est définie par $f(y)d\nu(y)$, 
où la densité $f: \mathbb{N} \to (0, \infty)$ est donnée par $f(y) = \frac{\lambda^y}{y!}e^{-\lambda}$
pour tout $y \in \mathbb{N}$. Écrire $\lambda = e^{\theta}$ pour un $\theta \in \mathbb{R}$. On a alors:

\begin{equation*}
    f(y) = \exp{- \lambda + y \ln {\lambda} - \ln {y!}} = \exp{\theta y - e^{\theta} - \ln {y!}}
\end{equation*}

Par conséquent, la loi de Poisson appartient à la famille exponentielle avec l'espace paramétrique naturelle $\mathbb{R}$.
Sous la forme $f(y) = \exp{\frac{y \theta - b(\theta)}{a(\phi)} + c(y, \phi)}$, 
on a que $a(\phi) \equiv 1$, $b(\theta) = e^{\theta}$ et $c(y, \phi) = - \ln {y!}$.////

\end{CJK*}
\end{document}