\documentclass[../main.tex]{subfiles}

\begin{document}
\section{Exercice 3}
\begin{CJK*}{UTF8}{gbsn}
Fix $\alpha > 0$. For a $\theta \in (0, \alpha)$,
let $\{X_i\}_{i=1}^n$ be a sequence of real independent identically distributed random variable 
defined on some probability space $(\Omega, F, P)$
with common probability density function with respect to the Lebesgue measure:

\begin{equation*}
    f_{\theta}(x) = 
    \begin{cases}
        \frac{2x}{\alpha \theta} \tab[1.3cm] x \in [0, \theta] \\
        \frac{2(\alpha-x)}{\alpha(\alpha - \theta)} \tab[0.6cm] x \in [\theta, \alpha] \\
        0 \tab[1.5cm] \text{otherwise}
    \end{cases}
\end{equation*}

Prove that the maximum likelihood estimation of $\theta$ must be one of the given observation
but not necessarily any particular observation. 
In case $\alpha = 5$ and $n=3$, compute the maximum likelihood estimate of $\theta$ when the 
observations are $(1,2,4)$ or $(2,3,4)$.  

\smallskip
\paragraph{Solution}
Let $x = (x_1, \cdots, x_n) \in \mathbb{R}^n$ be given. 
Write $g_{\theta}(x)$ for the joint probability density function of $X_i$'s 
and $x_{(i)}$ for the $i$th smallest coordiante in $x$.
If $x_i < 0$ or $x_i > \alpha$ for some $i$, then $g_{\theta}(x)=0$ for any $\theta$ so that 
any estimate would be a maximum likelihood estimate.
We exclude this pathology and prove that:

\begin{theorem}
If $0 \leqslant x_{(1)} \leqslant \cdots \leqslant x_{(n)} \leqslant \alpha$, then $\theta_0 = x_{(i)}$ for some $i$.
\end{theorem}

\begin{proof}
It never happens that $\theta_0 < x_{(1)}$
because $g_{\theta_1}(x) > g_{\theta_0}(x)$ whenever $\theta_1 \in (\theta_0,x_{(1)})$.
Similarly, it never happens that $\theta_0 > x_{(n)}$ because
$g_{\theta_1}(x) > g_{\theta_0}(x)$ whenever $\theta_1 \in (x_{(n)}, \theta_0)$.
We assume from now on that $\theta_0 \in [x_{(i)}, x_{(i+1)}]$ for some $i \in \{1,\cdots, n-1\}$.
Suppose, for the sake of contradiction, that $x_{(i)} < \theta_0 < x_{(i+1)}$. We have:

\begin{equation*}
    g_{\theta_{\theta_0}}(x) = \bracket{\frac{2}{\alpha}}^n  \frac{x_{(1)}}{\theta_0} \cdots \frac{x_{(i)}}{\theta_0}
    \frac{\alpha-x_{(i+1)}}{\alpha - \theta_0} \cdots \frac{\alpha-x_{(n)}}{\theta_0}
\end{equation*}

The numerator does not depend on $\theta_0$.
This motivates us to define function $h: [x_{(i)}, x_{(i+1)}] \to \mathbb{R}$ by:

\begin{equation*}
    h(\theta) = \frac{1}{\theta^i} \frac{1}{(\alpha-\theta)^{n-i}}
\end{equation*}

Then the second derivative is:

\begin{equation*}
    h''(\theta) = i(i+1)\theta^{-i-2} (\alpha-\theta)^{i-n} + (n-i)(n-i+1)\theta^{-i} (\alpha-\theta)^{i-n-2} > 0
\end{equation*}

Therefore, $h$ is strictly convex and the maximum can only be at the boundary points.
\end{proof}

We now demonstrate that the choice of $i$ is not unique in the above theorem. 
The simplest case will be $x_i=x_j$ for any $i$ and any $j$. 
For a nontrivial example, let $\alpha = 5$, $n = 3$.
If $x=(2,3,4)$, then the maximum likelihood estimate is one of $\{2,3,4\}$.
An estimate of $3$ or $4$ yields maximum likelihood $\frac{8}{375}$ while an estimate of $2$ yields likelihood $\frac{16}{1125}$.
Therefore, the maximum likelihood estimate can be $3$ or $4$ and is not unique.

Finally, the additional example $x=(1,2,4)$, estimate $\theta=1,2,4$ gives likelihood $\frac{3}{250}, \frac{4}{375}, \frac{1}{125}$
repsectively. We conclude that $\theta=1$ is the maximum likelihood estimate in this case.////
\end{CJK*}
\end{document}