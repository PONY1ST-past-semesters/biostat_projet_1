\documentclass[../main.tex]{subfiles}

\begin{document}
\section*{Exercice 1b}
\begin{CJK*}{UTF8}{gbsn}
Afin de faire la régression logistique, soit $Y$, $X_1$, $X_2$ et $X_3$ quatre 
variables aléatoires à valeur $\mathbb{R}$ définies sur 
l'espace de probabilité commune $(\Omega, \mathcal{F}, \mathbb{P})$ telles que 
$Y \in \{0,1\}$, $X_1 \in \{0,1\}$ et $X_2$ sont variables binaires catégoriels et $X_3$ est continue.
Supposons qu'il y a quatre constantes réelles $\beta_0$, $\beta_1$, $\beta_2$ et $\beta_3$ telles que :

\begin{equation*}
    P \{Y = 1 \mid X_1 = x_1, X_2 = x_2, X_3 = x_3 \} = \pi_1(x_1,x_2,x_3) = \frac{1}{1 + \exp{-(\beta_0 + \beta_1 x_1 + \beta_2 x_2 + \beta_3 x_3)}}
\end{equation*}

Assumer premièrement que $X_2$ et $X_3$ sont constantes presque partout, montrez que $\beta_1$ peut 
être interprété comme un log-rapport de cotes. 

Dans le deuxième cas, trouvez la différence du log-rapport de cotes pour deux individus.
Le premier individu a $X_1=1$ et $X_3 = 7$, le deuxième individu a $X_1=0$ et $X_3 = 5$, et les deux 
individus ont la même valeur de $X_2$.

\smallskip
\paragraph{Solution}
Si $f: \mathbb{R} \to (1, \infty)$ et $f(x) = \frac{1}{1+e^{-x}}$, alors $f$ est une bijection parce que la fonction exponentielle est une bijection.
La fonction inverse de $f$ est $f^{-1}: (1, \infty) \to \mathbb{R}$ et $f^{-1}(y) = \ln{\frac{y}{1-y}}$.

Supposons que $X_2 = x_2$ et $X_3 = x_3$ presque partout. On a :

\begin{equation*}
    \beta_0 + \beta_1 X_1 + \beta_2 x_2 + \beta_3 x_3 = \ln{\frac{ \pi_1(X_1,x_2,x_3)}{1-\pi_1(X_1,x_2,x_3)}}
\end{equation*}

Si $X_1 = 1$, on a:

\begin{equation*}
    \beta_1= \ln{\frac{ \pi_1(1,x_2,x_3)}{1-\pi_1(1,x_2,x_3)}} - \beta_2 x_2 - \beta_3 x_3 - \beta_0
\end{equation*}

Donc, $\beta_1$ peut être interprété comme un log-rapport de cotes.
Dans le deuxième cas, supposons que $X_2 = x_2$ pour tous les deux individus, alors on a :

\begin{equation*}
    \ln \frac{ \frac{\pi_1(1,x_2,7)}{1-\pi_1(1,x_2,7)}}{ \frac{\pi_1(0,x_2,5)}{1-\pi_1(0,x_2,5)}} =  
    \ln \frac{\pi_1(1,x_2,7)}{1-\pi_1(1,x_2,7)} - \ln \frac{\pi_1(0,x_2,5)}{1-\pi_1(0,x_2,5)} =  \beta_1 + 2\beta_3
\end{equation*}

Le calcul est complet.////
\end{CJK*}
\end{document}