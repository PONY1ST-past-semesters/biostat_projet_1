\documentclass{article}
\usepackage{graphicx} % Required for inserting images
\usepackage[utf8]{inputenc}
\usepackage[T1]{fontenc}
\usepackage{amsmath, amssymb, bm}


\begin{document}

\section*{0.4 Exercice 1d}

Les modèles linéaires généralisés (GLM) sont formulés comme :

\[ g(E[Y]) = \mathbf{X}\beta \]

où,
\begin{itemize}
    \item \( E[Y] \) est la valeur attendue de la variable de réponse \( Y \).
    \item \( g \) est la fonction de lien.
    \item \( \mathbf{X} \) est la matrice de conception.
    \item \( \beta \) est le vecteur de paramètres à estimer.
\end{itemize}

Considérons les composants principaux des GLM :

1. \textbf{Fonction de lien et moyenne} :
Les GLM supposent qu'une fonction de la variable de réponse (souvent la moyenne) est une combinaison linéaire des variables prédictives. Par exemple, dans la régression logistique, la log-vraisemblance de la réponse est une fonction linéaire des variables prédictives :

\[ \log\left(\frac{p}{1-p}\right) = \mathbf{X}\beta \]

où \( p = E[Y] \) est la probabilité que la réponse soit 1.

2. \textbf{Fonction de vraisemblance} :
Les paramètres du modèle sont estimés en maximisant la fonction de vraisemblance. Pour un ensemble de données observées, la fonction de vraisemblance est donnée par :

\[ L(\beta) = f(y | \mathbf{X}, \beta) \]

où \( f \) est la fonction de densité de probabilité des données pour un ensemble donné de paramètres \( \beta \). Notez que nous nous soucions uniquement de la valeur de la densité de probabilité à ce moment précis, et non de la distribution complète de \( Y \).

3. \textbf{Pourquoi une distribution complète n'est-elle pas nécessaire ?}
Notre objectif est de trouver les valeurs des paramètres qui maximisent la fonction de vraisemblance. Nous devons donc uniquement savoir quelle est la vraisemblance des données observées pour un ensemble donné de paramètres, sans nous soucier du reste de la distribution. Nous n'avons pas besoin de connaître la forme entière de la distribution, mais seulement comment la fonction de densité ou de masse de probabilité se comporte pour les données observées sous un ensemble donné de paramètres.

En conclusion, l'estimation des paramètres dans les GLM se concentre principalement sur la vraisemblance des données sous un ensemble donné de paramètres, et non sur la distribution complète de \( Y \).

\end{document}
