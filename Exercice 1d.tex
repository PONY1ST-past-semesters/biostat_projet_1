\documentclass[../main.tex]{subfiles}

\begin{document}
\begin{CJK*}{UTF8}{gbsn}
\section*{Exercice 1d}
Expliquez assez brièvement, pourquoi les modèles de régression linéaire généralisés
ne nécessitent pas réellement la spécification complète de la distribution de
la variable de réponse pour l’estimation des coefficients.

\paragraph{Solution}
Pour une mesure $\sigma$-finie $\nu$ définie sur une espace mesurable $(\Omega, F)$,
une famille $\mathcal{P} = \{f_{\theta}d\nu : \theta \in \Theta \subseteq \mathbb{R}\}$ continue 
absolument est appelé une famille exponentiel si, 
dans une forme plus simple et comme dans les notes de cours, on a pour tout $x \in \Omega$ que :

\begin{equation*}
    f_{\theta}(x) = \exp\curly{\frac{x\theta -b(\theta)}{a(\phi)} + c(x, \phi)}
\end{equation*}

Ici $\phi > 0$ est un paramètre de dispersion, $a : \mathbb{R} \to \mathbb{R}$,
$b : \mathbb{R} \to \mathbb{R}$ et $c : \Omega \times \mathbb{R} \to \mathbb{R}$ 
sont des fonctions telles que $a \neq 0$, $b$ est trois fois différentiables et $b'' > 0$.
On obtient les modèles linéaires généralisés si on utilise $\theta = g(\eta)$ pour 
une fonction de lien $g: \mathbb{R} \to \mathbb{R}$ et $\eta = \langle x , \beta \rangle_{\mathbb{R}^p}$ 
pour un vector constante $x \in \mathbb{R}^p$ et les paramètres $\beta \in \mathbb{R}^p$.
Soit $Y$ une réalisation de $f_{\theta} d \nu \in \mathcal{P}$.
Comme dans le notes de cours, on a que :

\begin{equation*}
    \mathbb{E}[Y] = b'(\theta); \tab[0.5cm] \text{Var}(Y) = b''(\theta) a(\phi)
\end{equation*}

On observe que la variance est une fonction de moyen, c'est-à-dire, on peut tout simpliment
dériver le moyen afin d'obtenir la variance. En effet, si $Y$ a plus de moments, 
les propriétés de la famille exponentiel ditent que tout les moments peuvent être dérivés
en fonction du moyen. Si $Y$ a une fonction génératrice des moments, 
alors sa fonction génératrice des moments est aussi contrôlé par le moyen.
En conclusion, contrôler le moyen égale contrôler la distribution si $Y$ est suffisament "régulière".
Le pramètre $\beta$, dans un autre côté, ne dépend que le moment. 
Il s'agit toujours d'estimer le moment et puis utiliser le moment pour estimer $\beta$
en faisant les inversions des fonctions ou matrices, sans spécifier la distribution de $Y$.////

\end{CJK*}
\end{document}
