\documentclass[../main.tex]{subfiles}

\begin{document}
\section*{Exercice 1c}
\begin{CJK*}{UTF8}{gbsn}
Soit $X$, $Y$ et $Z$ trois variables aléatoires à valeur $\mathbb{R}$ définies sur
l'espace de probabilité commune $(\Omega, \mathcal{F}, P)$ telles que :

\begin{enumerate}
    \item Pour tout $\omega \in \Omega$, la probabilité conditionnelle régulière $\mu_{Y \mid X, Z}(\omega, \cdot)$ 
    est la loi de Poisson avec paramètre:

    \begin{equation*}
        \exp(\beta_0 + \beta_1 X(\omega) + \beta_2 Z(\omega))
    \end{equation*} 

    Où $\beta_0$, $\beta_1$ et $\beta_2$ sont des constantes réelles.

    \item On a :
    
    \begin{equation*}
        \mathbb{E}[Y \mid X] = \exp{\beta_0+\beta_1 X} \mathbb{E}[e^{\beta_2 Z} \mid X]
    \end{equation*}
\end{enumerate}

Calculer $\text{Var}(Y \mid X)$.

\smallskip
\paragraph{Solution}
On a, pour chaque $\omega \in \Omega$ :

\begin{equation*}
    \mathbb{E}[Y^2 \mid X, Z](\omega) = \int_{\mathbb{R}} y^2 d \mu_{Y \mid X = X(\omega), Z = Z(\omega)} (y) = 
    \exp{2(\beta_0 + \beta_1X(\omega) + \beta_2Z(\omega))}  + \exp{\beta_0 + \beta_1X(\omega) + \beta_2Z(\omega)}
\end{equation*}

Alors :

\begin{equation*}
    \mathbb{E}[Y^2 \mid X] = \mathbb{E}[\mathbb{E}[Y^2 \mid X, Z] \mid X] = 
    \exp{2(\beta_0+\beta_1X)}\mathbb{E}[e^{2 \beta_2 Z} \mid X] + \exp{\beta_0 + \beta_1 X} \mathbb{E}[e^{\beta_2Z} \mid X]
\end{equation*}

Donc, en utilisant la formule de la variance conditionnelle :

\begin{equation*}
    \begin{split}
    &\tab[0.4cm] \text{Var}(Y \mid X) = \mathbb{E}[Y^2 \mid X] - \mathbb{E}[Y \mid X]^2 \\ & = 
    \exp{2(\beta_0+\beta_1X)}\mathbb{E}[e^{2 \beta_2 Z} \mid X] + \exp{\beta_0 + \beta_1 X} \mathbb{E}[e^{\beta_2Z} \mid X] 
    - \bracket{\exp{\beta_0+\beta_1 X} \mathbb{E}[e^{\beta_2 Z} \mid X]}^2 \\ & = 
    \exp{2(\beta_0+\beta_1X)} \text{Var}(e^{\beta_2 Z} \mid X) + \exp{\beta_0 + \beta_1 X} \mathbb{E}[e^{\beta_2Z} \mid X]
    \end{split}
\end{equation*}

Le calcul est complet. ////
\end{CJK*}
\end{document}